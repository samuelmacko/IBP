% Tento soubor nahraďte vlastním souborem s přílohami (nadpisy níže jsou pouze pro příklad)
% This file should be replaced with your file with an appendices (headings below are examples only)

% Umístění obsahu paměťového média do příloh je vhodné konzultovat s vedoucím
% Placing of table of contents of the memory media here should be consulted with a supervisor
%\chapter{Obsah přiloženého paměťového média}

%\chapter{Manuál}

%\chapter{Konfigurační soubor} % Configuration file

%\chapter{RelaxNG Schéma konfiguračního souboru} % Scheme of RelaxNG configuration file

%\chapter{Plakát} % poster

\chapter{Building and usage}

\section{Introduction}
\emph{ovirt-inventory} is an application for easy access to the RHEV inventory.

\section{Building}
To run \emph{ovirt-inventory}, user needs python3 and ovirt-engine repository already installed.

\begin{enumerate}
\item User can clone git repository containing source codes from \cite{git}.

In that case, user also needs to install python3-qt5 and python3-ovirt-engine-sdk4.
\item User can use a RPM file \cite{git} which will install everything needed.
\end{enumerate}

\section{Usage}
\subsection{Logging in}
After launch, input dialog containing three input fields is displayed.
\begin{itemize}
\item \emph{username} field expecting username and domain in format: username@domain
\item \emph{password} field expecting corresponding password
\item \emph{url} field expecting FQDN of the target virtual machine
\end{itemize}

\subsection{Hide/show columns}
By checking or unchecking checkboxes in 'Select Column', drop-down menu user can hide or show table columns.

\subsection{Refresh}
By pressing 'Refresh Button' user can load up-to-date data.

\subsection{Filter}
User can write custom filters into an filter field. Filter needs to be in format 'column name' '<|>|=' 'value'. Filter supports evaluating multiple filters at the same time. All sub-filters are in 'AND' relation and need to be separated from each other by ','.

\subsection{Ordering}
User can order items in columns in ascending and descending order by clicking on column he wants to order.

\subsection{Redirecting}
Some table cells contain multiple values (for example one virtual machine can contain multiple disks). In that case, the cell displays first item followed by the number in parentheses expressing how many other items are in that cell. By double clicking such cell, user is redirected to the corresponding tab and right filter is applied.

\subsection{Export to .csv}
User can export current table into .csv file. Go to 'File' -> 'export'.

\subsection{Save}
User can save current configuration of all tables into config file by going 'File' -> 'export'. These information will be saved in a config file:

\begin{itemize}
\item user name
\item domain
\item checkboxes from all the tables
\end{itemize}
